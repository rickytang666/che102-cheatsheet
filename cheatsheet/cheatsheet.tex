\documentclass[10pt,landscape,letterpaper]{article}
\usepackage{multicol}
\usepackage{calc}
\usepackage{ifthen}
\usepackage[landscape,letterpaper]{geometry}
\usepackage{amsmath,amsthm,amsfonts,amssymb}
\usepackage{color,graphicx,overpic}
\usepackage{hyperref}
\usepackage[version=4]{mhchem}
\usepackage{framed}
\usepackage{ulem}

\pdfinfo{
  /Title (che102 cheat sheet)
  /Creator (TeX)
  /Producer (pdfTeX 1.40.0)
  /Author (Ricky Tang, edited by Sean Yang)
  /Subject (Chemistry)
}

% Super narrow margins
\ifthenelse{\lengthtest { \paperwidth = 11in}}
    { \geometry{top=.3in,left=.3in,right=.3in,bottom=.3in} }
    {\ifthenelse{ \lengthtest{ \paperwidth = 297mm}}
        {\geometry{top=0.5cm,left=0.5cm,right=0.5cm,bottom=0.5cm} }
        {\geometry{top=0.5cm,left=0.5cm,right=0.5cm,bottom=0.5cm} }
    }

% Turn off header and footer
\pagestyle{empty}

% Redefine section commands to use less space
\makeatletter
\renewcommand{\section}{\@startsection{section}{1}{0mm}%
                                {-1.5ex plus -.5ex minus -.2ex}%
                                {1.5ex plus .3ex}%
                                {\normalfont\Large\bfseries\underline}}
\renewcommand{\subsection}{\@startsection{subsection}{2}{0mm}%
                                {-1.2ex plus -.5ex minus -.2ex}%
                                {1.2ex plus .3ex}%
                                {\normalfont\large\bfseries}}
\renewcommand{\subsubsection}{\@startsection{subsubsection}{3}{0mm}%
                                {-1ex plus -.4ex minus -.15ex}%
                                {1ex plus .2ex}%
                                {\normalfont\normalsize\bfseries}}
\makeatother

% Define BibTeX command
\def\BibTeX{{\rm B\kern-.05em{\sc i\kern-.025em b}\kern-.08em
    T\kern-.1667em\lower.7ex\hbox{E}\kern-.125emX}}

% Don't print section numbers
\setcounter{secnumdepth}{0}

\setlength{\parindent}{0pt}
\setlength{\parskip}{1.5pt plus 0.5ex}
\makeatother

% Reduce line spacing
\linespread{0.85}
\setlength{\FrameSep}{2pt}
\setlength{\OuterFrameSep}{1.5pt}

%My Environments
\newtheorem{example}[section]{Example}
% -----------------------------------------------------------------------

\begin{document}
\raggedright
\footnotesize
\begin{multicols}{4}

  % multicol parameters
  % These lengths are set only within the two main columns
  %\setlength{\columnseprule}{0.25pt}
  \setlength{\premulticols}{1pt}
  \setlength{\postmulticols}{1pt}
  \setlength{\multicolsep}{1pt}
  \setlength{\columnsep}{2pt}

  \section{pre-midterm stuff}

  \subsection{fundamentals}

  \textbf{$\Delta$T conversion factor: } $\displaystyle \frac{1K}{1.8F}$ \& $\displaystyle \frac{1^\circ C}{1^\circ F}$ 

  \textbf{molality: } $\displaystyle m = \frac{\text{mol solute}}{\text{kg solvent}}$

  \textbf{ppm and ppb:}
  $\displaystyle y_i = \text{ppm}_i \times 10^{-6} = \text{ppb}_i \times 10^{-9}$

  \subsection{gases}

  \textbf{ideal:} $\displaystyle PV=nRT$; $\displaystyle \rho = \frac{PM}{RT}$

  \textbf{real: van der waals equation}
  \begin{itemize}
    \item ideal behavior at high T, low P
  \end{itemize}

  \textbf{manometer:} $P = \rho g h$. watch units!

  \subsection{intermolecular forces}

  \textbf{strength:} H-bond $>$ Dipole-Dipole $>$ LDF
  \begin{itemize}
     \item \textbf{LDF:} all molecules. $\uparrow$ strength with size/mass (polarizability)
     \item \textbf{H-bond:} H bonded to N, O, F
  \end{itemize}

  \subsection{limiting reactant (LR)}

  \begin{enumerate}
    \item balance rxn
    \item calculate $n$ for all reactants
    \item divide them by coefficients and find min
  \end{enumerate}

  \section{rates of reaction}

  \subsection{collision theory}

  \textbf{conditions for rxn}: collide + sufficient $E_a$ + correct orientation

  \textbf{catalyst:}
  \begin{itemize}
    \item increases rate via \textbf{lower $E_a$ path}
    \item participates but \textbf{not altered} by rxn
    \item does \textbf{not appear} in overall rxn
    \item does \textbf{NOT} change equilibrium
  \end{itemize}

  \textbf{factors affecting rxn rate:}
  \begin{itemize}
    \item concentration of reactants
    \item temperature
    \item presence of \textbf{catalysts}
    \item physical nature of reactants
  \end{itemize}

  \textbf{$E_p$ diagrams}: enthalpy of reactants and products are \textbf{FIXED} regardless of catalyst presence.

  \subsection{reaction rates}

  must be determined \textbf{experimentally}

  rxn rate can get the rate of change of single species using \textbf{mole ratio}

  $\displaystyle \dot{R}=-\frac{\dot{R_{A}}}{a}\implies \dot{R}_{A}=-a\times \dot{R}$

  unit: $M/s$ ($\frac{mol}{L \cdot s}$)

  \textbf{positive for products, negative for reactants}

  \subsection{rate laws}

  relates rate to \textbf{concentration} of reactants

  \subsubsection{differential}

  for $aA+bB \rightarrow \text{products}$:
  $\dot{R}=-\frac{1}{a}\frac{d[A]}{dt}=-\frac{1}{b}\frac{d[B]}{dt}=k[A]^{m}[B]^{n}$

  \begin{itemize}
    \item $m$: order w.r.t. A  ;  $n$: order w.r.t. B
    \item $m+n$: overall order
    \item $k$: rate constant (only changes with \textbf{T})
  \end{itemize}

  \textbf{note:} exponents $\neq$ coefficients

  \subsubsection{integrated \& half-life}

  for $aA \rightarrow \text{products}$:

  \textbf{zero-order:}
  \begin{itemize}
    \item $[A]=-akt+[A]_{0}$
    \item unit of $k$: $M/s$
    \item $t_{1/2}=\frac{[A]_{0}}{2ak}$
  \end{itemize}

  \textbf{first-order:}
  \begin{itemize}
    \item $\ln[A]=-akt+\ln[A]_{0}$
    \item unit of $k$: $1/s$
    \item $t_{1/2}=\frac{\ln~2}{ak}$
  \end{itemize}

  \textbf{second-order:}
  \begin{itemize}
    \item $\frac{1}{[A]}=akt+\frac{1}{[A]_{0}}$
    \item unit of $k$: $1/(M\cdot s)$
    \item $t_{1/2}=\frac{1}{ak[A]_{0}}$
  \end{itemize}

  \textbf{n-order ($n \neq 1$):}
  \begin{itemize}
    \item (see formula sheet)
    \item unit of $k$: $\frac{1}{\text{M}^{n-1} \cdot \text{s}}$
    \item $t_{1/2}=\frac{2^{n-1} - 1}{ak(n-1)[A]_0^{n-1}}$
  \end{itemize}

  \subsection{arrhenius equation}

  (see formula sheet)

  preferrably use two-point form
  
  use $R = 8.314$ J/(mol $\cdot$ K)

  \vfill\null
  \columnbreak

  \section{phase equilibrium}

  \subsection{phase equilibrium}

  \textbf{defn:} rate (forward) = rate (reverse)
  
  e.g. vapour-liquid equilibrium $\implies$ $\text{rate}_{\text{vap}} = \text{rate}_{\text{cond}}$

  \textbf{note:} phase $\neq$ state of matter

  \textbf{phase changes:}
  $\text{gas} \xrightarrow{\text{condensation}} \text{liquid} \xrightarrow{\text{solidification}} \text{solid}$
  $\text{gas} \xleftarrow{\text{vaporization}} \text{liquid} \xleftarrow{\text{fusion}} \text{solid}$
  $\text{gas} \xleftarrow[\text{deposition}]{\text{sublimation}} \text{solid}$

  $\Delta H_{\text{fus}} + \Delta H_{\text{vap}} = \Delta H_{\text{sub}}$

  \subsection{vapour pressure}

  \textbf{vapour pressure:} pressure of vapour at equilibrium
  
  weak IMF $\rightarrow$ lower $T_{\text{bp}}$ $\rightarrow$ higher vapour pressure (more volatile); and \textbf{vice versa}

  $$\boxed{P_{\text{vap}} = f(T, \text{type of liquid})}$$

  \textbf{evaporation vs. boiling:}
  \begin{itemize}
    \item \textbf{evaporation:} at surface, any temperature
    \item \textbf{boiling:} throughout liquid, specific temperature ($T_{\text{bp}}$)
  \end{itemize}

  \textbf{boiling point:} temperature where $P_{\text{vap}}(T_{\text{bp}}) = P$. 
  
  normal boiling point: $P_{\text{vap}} = 1$ atm

  \subsubsection{clausius-clapeyron equation}

  (see formula sheet, use $R=8.314$ J/(mol $\cdot$ K))

  relates vapour pressure to \textbf{temperature}

  can be used for \textbf{ANY phase change} as long as the right enthalpy is used

  \subsubsection{non-equilibrium stuff (humidity/saturation)}

  $\displaystyle\%\text{ saturation} = \frac{P_{A}}{P_{A}^{\text{vap}}(T)} \times 100\%$

  \textbf{dew point $T_{\text{dp}}$:} temperature where humid air reaches saturation

  condensation occurs at \textbf{100 \% saturation}

  humidity refers specifically to $\ce{H2O}$

  \subsection{phase diagrams}

  can be used to get the vapour pressure at a temperature (using the equilibrium lines)

  \textbf{triple point:} solid, liquid, gas coexist $\rightarrow$ there can be multiple triple points, but only one is solid-liquid-gas equilibrium

  \textbf{equilibrium line:} 2 phases in equilibrium

  \textbf{critical point:} point where substance becomes supercritical fluid

  \textbf{supercritical fluid:} neither liquid nor gas, but having properties of both

  \textbf{polymorphism:} existence of solid in more than one form

  \textbf{note:} a substance can phase change at multiple different temperatures or pressures, but at a given P (T), it phase changes at the corresponding T (P).

  \subsection{henry's law}

  \textbf{essentially:} gas \textbf{solubility} in liquid increases with increasing \textbf{pressure}

  \textbf{ideal solution:} $\Delta H_{\text{soln}} = 0$, similar forces between all components

  use $P_A = H_A x_A$ when constants have units of \textbf{pressure}, and working with \textbf{mole fractions}

  use $C_A = k_A P_A$ when constants have units of \textbf{pressure \& concentration}, and working with \textbf{concentrations}

  \subsection{raoult's law}

  \textbf{essentially:} adding solute \textbf{lowers} the vapour pressure of solvent

  $P_A = x_A P_A^{\text{vap}}$ $\rightarrow$ use this if u are given vapour pressures

  applies to ideal solutions or dilute solutions ($x_{\text{solv}} > 0.98$)

  solute and solvent both in vapour and in solution

  \textbf{deviations:} positive if total pressure is greater than each individual pure vapour pressure (and vice versa)

  \subsection{colligative properties}

  \textbf{defn:} properties of solutions that depend on the \textbf{RATIO} of solute particles to solvent molecules (NOT type of solute)

  \subsubsection{vapour pressure lowering}

  (see formula sheet)

  we can use vapour pressure to estimate the \textbf{molar mass} of an unknown solid dissolved in a known liquid

  $$M_{\text{solid}}=-\frac{M_{\text{liquid}}m_{\text{solid}}}{m_{\text{liquid}}}\left( \frac{P^{\text{vap}}_{\text{liquid}}}{\Delta P_{\text{liquid}}}+1 \right)$$

  \subsubsection{b.p. elevation \& f.p. depression}

  $\Delta T_{\text{bp}} = i K_b m$, $\Delta T_{\text{fp}} = -i K_f m$

  $m$ is the solute \textbf{MOLALITY (moles solute/kg solvent)}!

  $K_b$ and $K_f$ are constants, dependent on the \textbf{solvent} only

  \textbf{van't hoff factor $i$} $\rightarrow$ for ionic compounds 
  
  $i_{\text{min}} = 1$ (pure solid/liquid, no dissociation)

  $i_{\text{max}} = \text{\# of ions}$ (complete dissociation), e.g. 2 for \ce{NaCl}

  $\displaystyle\boxed{\text{\% dissociation} = \frac{i - 1}{i_{\text{max}} - 1} \times 100\%}$

  \columnbreak

  \section{chemical equilibrium}

  \textbf{chemical equilibrium:} at equilibrium, forward rate = reverse rate

  concentrations at equilibrium stay constant, but the reaction is \textbf{still going on}

  \subsection{equilibrium constants}

  \begin{framed}
    depends on \textbf{temperature ONLY}
  \end{framed}

  for $aA + bB \rightleftharpoons cC + dD$:

  $$K_c = \frac{[C]^c[D]^d}{[A]^a[B]^b} \quad K_P = \frac{P_C^c P_D^d}{P_A^a P_B^b}$$

  \begin{framed}
    for pressure use units of \textbf{bar}, and $R = 0.08314$ L$\cdot$bar/(mol$\cdot$K)
  \end{framed}

  only include aqueous and gaseous species, \textbf{do NOT include pure liquids and solids}

  \textbf{relationship:} 
  $$K_P = K_c(RT)^{\Delta n} \quad \text{or} \quad K_c = K_P(RT)^{-\Delta n}$$
  where $\Delta n = (c + d) - (a + b)$

  \textbf{magnitude of $K$:}
  \begin{itemize}
    \item $K > 10^{10}$: reaction goes to completion
    \item $K < 10^{-10}$: reaction does not occur forward
  \end{itemize}

  \subsubsection{properties of equilibrium constants}

  \begin{itemize}
    \item multiply reaction by constant $\rightarrow$ raise $K$ to the power of that constant
    \item $\displaystyle K_{\text{reverse}} = \frac{1}{K_{\text{forward}}}$
    \item add reactions: \textbf{multiply} equilibrium constants
  \end{itemize}

  \subsubsection{equilibrium constant calculations}

  usually use ICE table approach

  \subsection{reaction quotient}

  same formula as $K$ but use the data at ANY point (not necessarily equilibrium)

  \textbf{direction of change:}
  \begin{itemize}
    \item $Q_c < K_c$: reactants excess, forward reaction
    \item $Q_c > K_c$: products excess, reverse reaction
  \end{itemize}

  \subsection{le chatelier's principle}

  \subsubsection{concentration changes}

  \begin{itemize}
    \item add reactant: forward reaction, more products
    \item add product: reverse reaction, more reactants
  \end{itemize}

  \subsubsection{volume/pressure changes}

  \begin{framed}
    if there're SAME number of gas molecules on both sides, changing volume/pressure does NOT affect equilibrium
  \end{framed}

  recall: when T is constant, $V \propto \frac{1}{P}$

  \begin{itemize}
    \item reduce volume (increase pressure): shifts towards side with fewer gas molecules
    \item increase volume (decrease pressure): shifts towards side with more gas molecules
  \end{itemize}

  \subsubsection{temperature changes}

  \begin{itemize}
    \item endothermic ($\Delta H > 0$): $T \uparrow$, K increases, shifts to products, and vice versa
    \item exothermic ($\Delta H < 0$): $T \uparrow$, K decreases, shifts to reactants, and vice versa;
  \end{itemize}

  determine K at a certain temperature (two point form) $\rightarrow$ \textbf{van't hoff equation} (see formula sheet, use $R=8.314$ J/(mol$\cdot$K))

  \begin{framed}
    adding solids, liquids or inert gases does NOT affect equilibrium, \textbf{unless it causes a change of concentration, P, T or V}
  \end{framed}

  \textbf{inert gas addition:}
  \begin{itemize}
    \item @ const. \textbf{volume}: $P_{\text{total}}$ increases, but $P_{i}$ unchanged $\rightarrow$ \textbf{NO shift}.
    \item @ const. \textbf{pressure}: $V_{\text{total}}$ increases, $P_{i}$ decreases $\rightarrow$ shifts to \textbf{more gas moles}.
  \end{itemize}

  \begin{framed}
    when one or multiple changes occur and you’re unsure which way the reaction will go, use Q!
  \end{framed}

  \vfill\null
  \columnbreak
  
  \section{electrochemistry}

  \subsection{oxidation states \& redox rxns}

  \textbf{rules for assigning OS:}
  \begin{itemize}
    \item free element: OS = 0 (e.g. \ce{H2}, \ce{O2})
    \item monatomic ion: OS = charge (e.g. \ce{Na+})
    \item sum of OS in neutral species = 0
    \item sum of OS in polyatomic ion = charge
    \item H is generally +1, O is generally -2
    \item others: \textit{USUALLY} just check group \#
    \item \textbf{exceptions:} H is -1 in hydrides (e.g., \ce{LiH}); O is -1 in peroxides (e.g., \ce{H2O2})
  \end{itemize}

  \textbf{redox reactions:}
  \begin{itemize}
    \item reduction: OS decreases, gains electrons
    \item oxidation: OS increases, loses electrons
    \item reducing agent: stuff that gets oxidized
    \item oxidizing agent: stuff that gets reduced
  \end{itemize}

  \textbf{notes:} to become redox rxn, there needs to be a reduction and an oxidation (OS changes)

  \textbf{balancing redox rxns:}
  \begin{enumerate}
    \item write separate half-reactions for oxidation and reduction
    \item balance all atoms except H, O
    \item balance O with \ce{H2O}
    \item balance H with \ce{H+}
    \item balance charges with $e^-$
    \item multiply \& add half-rxns to cancel $e^-$
    \item for basic: add \ce{OH-} equal to \ce{H+} to each side, simplify (combine \ce{H+} and \ce{OH-} to form \ce{H2O})
  \end{enumerate}

  \subsection{galvanic cell \& faraday's law}

  \textbf{galvanic cell:} spontaneous redox rxns $\rightarrow$ electrical energy

  notation: $\text{Zn}(s)|\text{Zn}^{2+}(aq)||\text{Cu}^{2+}(aq)|\text{Cu}(s)$

  \textbf{anode (oxidation):} lower potential, electrons flow from anode. 

  \textbf{cathode (reduction):} higher potential, electrons flow to cathode

  \textbf{current \& faraday's law:} $\displaystyle n = \frac{Q}{F} = \frac{It}{F}$

  where $F = 96485$ C/mol (charge of 1 mole of electrons), $n$ = moles of electrons

  \textbf{current efficiency:}

  $\displaystyle \eta = \frac{\text{charge used}}{\text{charge supplied}} = \frac{\text{act. mass}}{\text{theor. mass}} \times 100\%$

  \subsection{standard cell potential}

  \textbf{std. conditions:} 1.0 M (aqueous), 1 bar (gases), 25$^\circ$C

  $$E^\circ_{\text{cell}} = E^\circ_{\text{redu.}} - E^\circ_{\text{oxid.}} = E^\circ_{\text{cathode}} - E^\circ_{\text{anode}}$$

  \textbf{spontaneous direction:} electrons travel from low to high potential. higher $E^\circ$ gets reduced (cathode). \textbf{$E^\circ_{\text{cell}} > 0$ for spontaneous}

  \textbf{non-standard conditions $\rightarrow$ nernst equation} (see formula sheet)

  \begin{framed}
    use $n=$ \# electrons transferred, $R = 8.314\;J/(\text{mol}\cdot \text{K})$
  \end{framed}

  \textbf{nernst @ 25$^\circ$C:}
  $\displaystyle E_{\text{cell}} = E^\circ_{\text{cell}} - \frac{0.0592}{n}\log Q$

  \textbf{concentration cell:} same material at anode and cathode. \textbf{higher concentration} acts as \textbf{cathode}, $E^\circ_{\text{cell}} = 0$

  \subsubsection{general 3 first steps for galvanic cell problems}

  \begin{enumerate}
    \item identify oxidation and reduction half-reactions using the reduction potentials
    \item add them up (with same \# of \ce{e-}) to get overall reaction
    \item calculate $E_{\text{cell}}$
  \end{enumerate}

  \subsubsection{find best oxidizing/reducing agent}

  \textbf{best oxid. agent:} choose \textbf{highest} $E^\circ$

  \textbf{best redu. agent:} choose \textbf{lowest} $E^\circ$

  \subsubsection{nernst at equilibrium}

  battery dies when:

  \begin{itemize}
    \item equilibrium is reached ($E_{\text{cell}} = 0$, $Q = K$)
    $\displaystyle \boxed{\displaystyle K = \exp\left(\frac{E^\circ_{\text{cell}}nF}{RT}\right)}$
    
    if $K>e^{10}$ or $K<e^{-10}$, the rxn approx. fully goes in the right/left direction
    \item the LR runs out (not in equilibrium)
  \end{itemize}

  \subsection{electrochemical cells}

  \textbf{galvanic vs. electrolytic:}
  \begin{itemize}
    \item galvanic: derives electrical energy from spontaneous redox ($E_{\text{cell}} > 0$)
    \item electrolytic: uses electrical energy to promote non-spontaneous reaction ($E_{\text{cell}} < 0$)
  \end{itemize}

  \textbf{electrochemical cells:}
  \begin{itemize}
    \item anode: always oxidation
      \begin{itemize}
        \item galvanic (negative): \ce{e-} are freed by the oxidation half-reaction
        \item electrolytic (positive): \ce{e-} are withdrawn from electrode
      \end{itemize}
    \item cathode: always reduction
      \begin{itemize}
        \item galvanic (positive): \ce{e-} are removed by the reduction half-reaction
        \item electrolytic (negative): \ce{e-} are forced onto electrode
      \end{itemize}
  \end{itemize}

\end{multicols}
\end{document}
